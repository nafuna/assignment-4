\documentclass[]{article}
%opening
\title{GOOGLE DEVELOPER'S PRODUCT:\\GMAIL\\by}
\author{NAFUNA MERETH GAMBWA\\211007767}

\begin{document}

\maketitle

\section{ GMAIL}
The idea for Gmail was developed by Paul Buchheit several years before it was announced to the public. The project was known by the code name Caribou. During early development, the project was kept secret from most of Google's own engineers. This changed once the project became better and better, and by early 2004, almost everybody was using it to access the company's internal email system.[1]

Gmail was announced to the public by Google on April 1, 2004 as a limited beta release.[2] In November 2006, Google began offering a Java-based application of Gmail for mobile phones.[3]

In October 2007, Google began a process of rewriting parts of the code that Gmail used, which would make the service faster and add new features, such as custom keyboard shortcuts and the ability to bookmark specific messages and email searches.[4] Gmail also added IMAP support in October 2007.[5]

An update around January 2008 changed elements of Gmail's use of JavaScript, and resulted in the failure of a third-party script some users had been using. Google acknowledge the issue and helped users with workarounds.[6]

Gmail exited the beta status on July 7, 2009.[7] Prior to December 2013, users had to approve to see images in emails, which acted as a security measure. This changed in December 2013, when Google, citing improved image handling, enabled images to be visible without user approval. Images will be routed through Google's secure proxy servers rather than the original external host servers.[8] 

MarketingLand noted that the change to image handling means email marketers will no longer be able to track the recipient's IP address or information about what kind of device the recipient is using.[9] However, Wired stated that the new change means senders can track the time when an email is first opened, as the initial loading of the images requires the system to make a "callback" to the original server.[10]
\begin{thebibliography}{10}
\bibitem{Harry}McCracken, Harry (April 1, 2014). "How Gmail Happened: The Inside Story of Its Launch 10 Years Ago". Time. Time Inc. Retrieved march 6, 2018.
\bibitem{Google} "Google Gets the Message, Launches Gmail". Google. April 1, 2004. Retrieved march 6, 2018.
\bibitem{Oswald}Oswald, Ed (November 2006). "Google Offers Java-based Mobile Gmail". BetaNews. Retrieved march 6, 2018.
\bibitem{Pupius}Pupius, Dan (October 29, 2007). "Code changes to prepare Gmail for the future". Official Gmail Blog. Google. Retrieved march 6, 2018.
\bibitem{Jones}Jones, K.C. (October 24, 2007). "Gmail Now Has IMAP Support". InformationWeek. UBM Tech. Retrieved march 6, 2018.
\bibitem{Dan} Pupius, Dan (January 29, 2008). "Gmail/Greasemonkey API issue". Official Gmail Blog. Google. Retrieved march 6, 2018.
\bibitem{Matthew}Glotzbach, Matthew (July 7, 2009). "Google Apps is out of beta (yes, really)". Official Google Blog. Google. Retrieved march 6, 2018.
\bibitem{ Rae} Rae-Grant, John (December 12, 2013). "Images Now Showing". Official Gmail Blog. Google. Retrieved march 6, 2018.
\bibitem{Land} "Google: Gmail Image Change May Improve Open Rate Data, But Will Strip Other User Data". MarketingLand. December 12, 2013. Retrieved march 6, 2018.
\bibitem{Tate}Tate, Ryan (December 12, 2013). "With the New Gmail, People Will Know When You Open That Message". Wired. Condé Nast. Retrieved march 6, 2018.
\end{thebibliography}
\end{document}
